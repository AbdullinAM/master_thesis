%%%%%%%%%%%%%%%%%%%%%%%%%%%%%%%%%%%%%%%%%%%%%%%%%%%%%%%%%%%%%%%%%%%%%%%%%%%%%%%%
%%%%%%%%%%%%%%%%%%%%%%%%%%%%%%%%%%%%%%%%%%%%%%%%%%%%%%%%%%%%%%%%%%%%%%%%%%%%%%%%
\chapter{Постановка задачи объединения абстрактной интерпретации и метода 
ограниченной проверки моделей}
%%%%%%%%%%%%%%%%%%%%%%%%%%%%%%%%%%%%%%%%%%%%%%%%%%%%%%%%%%%%%%%%%%%%%%%%%%%%%%%%
%%%%%%%%%%%%%%%%%%%%%%%%%%%%%%%%%%%%%%%%%%%%%%%%%%%%%%%%%%%%%%%%%%%%%%%%%%%%%%%%
Для успешного проведения исследования, необходимо четко определить решаемые 
задачи. В данном разделе:
\begin{itemize}
\item формулируются задачи, решаемые во время исследования;
\item выбираются инструменты, на основе которых будет проводиться исследование;
\item предлагаются подходы к решению поставленных задач;
\item определяются требования к прототипу системы, на основе которой будет 
проводиться исследование.
\end{itemize}

%%%%%%%%%%%%%%%%%%%%%%%%%%%%%%%%%%%%%%%%%%%%%%%%%%%%%%%%%%%%%%%%%%%%%%%%%%%%%%%%
\section{Задача объединения методов абстрактной интерпретации и ограниченной 
проверки моделей}
%%%%%%%%%%%%%%%%%%%%%%%%%%%%%%%%%%%%%%%%%%%%%%%%%%%%%%%%%%%%%%%%%%%%%%%%%%%%%%%%
Задача данного исследования состоит в разработке подхода к объединению 
абстрактной интерпретации и метода ограниченной проверки моделей с целью 
повышения производительности анализа. Основная идея предлагаемого подхода 
заключается в том, что использование абстрактной интерпретации при предварительной
обработке программы позволит избежать лишних вызовов SMT решателя в BMC и,
таким образом, уменьшить время работы анализатора. Таким образом, AI должна
иметь высокую производительность, но при этом сохранять приемлемую 
точность~(иначе она не окажет никакого влияния на BMC).

Предлагается решать данную задачу на двух уровнях представления программы:
\begin{itemize}
\item на уровне исходного кода;
\item на уровне упрощенного представления программы, используемого в BMC.
\end{itemize}

При интерпретации на уровне исходного кода программа будет представлена в своем
исходном виде, значит в ней будут присутствовать циклы, рекурсии и 
межпроцедурность. С одной стороны, сохранение всех этих элементов программы 
дает анализу больше информации, то есть AI будет обладать большей точностью~(в
сравнении с AI упрощенного представления). С 
другой стороны, все эти элементы значительно увеличивают сложность анализа.

Абстрактная интерпретация на уровне упрощенного представления BMC имеет 
следующие свойства:
\begin{itemize}
\item AI обладает полнотой относительно BMC: при таком подходе семантика
упрощенного представления программы является \emph{конкретной семантикой};
\item в упрощенном представлении программы отсутствуют многие элементы 
программы, которые значительно усложняют анализ~(циклы, рекурсии и т.д.), то 
есть AI будет обладать большей производительностью~(в сравнении с AI исходной 
программы);
\item на данном уровне можно использовать некоторые дополнительные техники,
которые позволяют еще больше упростить программу или добавить в нее какую-либо
информацию~(контракты, слайсинг~\cite{slicing} и т.д.).
\end{itemize}

Решение описанных задач позволит оценить, насколько объединение AI и BMC 
применимо для анализа реальных программ.

%%%%%%%%%%%%%%%%%%%%%%%%%%%%%%%%%%%%%%%%%%%%%%%%%%%%%%%%%%%%%%%%%%%%%%%%%%%%%%%%
\section{Задача разработки модуля абстрактной интерпретации для системы 
Borealis}
%%%%%%%%%%%%%%%%%%%%%%%%%%%%%%%%%%%%%%%%%%%%%%%%%%%%%%%%%%%%%%%%%%%%%%%%%%%%%%%%
Borealis --- это система статического анализа языка С, основанная на методе 
ограниченной проверки моделей (Bounded Model Checking, BMC)~\cite{borealis}. Система 
построена на использовании компилятора Clang~\cite{clang} для разбора исходного 
кода, системы LLVM~\cite{llvm} для анализа кода и SMT решателей для поиска 
дефектов. Система способна искать дефекты двух типов: встроенные 
нефункциональные и заданные при помощи контрактов. Контракты задаются двумя 
способами: с помощью языка аннотаций, основанного на комментариях и схожего с 
языком ACSL~\cite{acsl}, а также с помощью встроенных в программный код вызовов специальных 
процедур. В данной работе рассматривается разработка модуля абстрактной 
интерпретации для системы Borealis. Данный модуль должен обладать следующими
свойствами:
\begin{itemize}
\item поддержка типов данных LLVM~(и, соответственно, Borealis);
\item поддержка всей семантики LLVM, которая может быть сгенерирована при 
анализе языка C;
\item поддержка всей семантики упрощенного представления программы в Borealis;
\item наличие инструментов проверки ошибок по результатам AI;
\item представление результатов работы интерпретатора в формате, понятном 
системе Borealis.
\end{itemize}

Язык разработки прототипа --- C++~\cite{languageC++}, так как система Borealis 
разработана на языке C++.

%%%%%%%%%%%%%%%%%%%%%%%%%%%%%%%%%%%%%%%%%%%%%%%%%%%%%%%%%%%%%%%%%%%%%%%%%%%%%%%%
\section{Резюме}
%%%%%%%%%%%%%%%%%%%%%%%%%%%%%%%%%%%%%%%%%%%%%%%%%%%%%%%%%%%%%%%%%%%%%%%%%%%%%%%%
В данном разделе поставлена задача разработки технологии объединения методов
абстрактной интерпретации и ограниченной проверки моделей, определены основные 
требования. Также поставлена задача разработки прототипа модуля абстрактной
интерпретации для системы статического анализа Borealis для проверки 
применимости разработанной технологии и проведения качественной и 
количественной оценки ее эффективности.