%%%%%%%%%%%%%%%%%%%%%%%%%%%%%%%%%%%%%%%%%%%%%%%%%%%%%%%%%%%%%%%%%%%%%%%%%%%%%%%
%%%%%%%%%%%%%%%%%%%%%%%%%%%%%%%%%%%%%%%%%%%%%%%%%%%%%%%%%%%%%%%%%%%%%%%%%%%%%%%
\chapter{Разработка алгоритма технологии объединения абстрактной интерпретации 
и метода ограниченной проверки моделей}
\label{chapter:algorithm}
%%%%%%%%%%%%%%%%%%%%%%%%%%%%%%%%%%%%%%%%%%%%%%%%%%%%%%%%%%%%%%%%%%%%%%%%%%%%%%%
%%%%%%%%%%%%%%%%%%%%%%%%%%%%%%%%%%%%%%%%%%%%%%%%%%%%%%%%%%%%%%%%%%%%%%%%%%%%%%%
В соответствии с поставленными задачами необходимо разработать технологию
объединения АИ и BMC. Технология состоих из нескольких этапов:
\begin{itemize}
\item интерпретация исходного кода программы;
\item проверка программы на наличие ошибок по результатам АИ;
\item интерпретация упрощенного представления программы;
\item проверка упрощенного представления программы на наличие ошибок по 
результатам АИ;
\end{itemize}

В данном разделе рассматриваются модели представления кода, для которых 
выполнялась разработка алгоритмов и описываются основные идеи, положенные в их
основу.