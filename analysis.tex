%%%%%%%%%%%%%%%%%%%%%%%%%%%%%%%%%%%%%%%%%%%%%%%%%%%%%%%%%%%%%%%%%%%%%%%%%%%%%%%
%%%%%%%%%%%%%%%%%%%%%%%%%%%%%%%%%%%%%%%%%%%%%%%%%%%%%%%%%%%%%%%%%%%%%%%%%%%%%%%
\chapter{Сравнительный анализ методов абстрактной интерпретации и ограниченной 
проверки моделей}
%%%%%%%%%%%%%%%%%%%%%%%%%%%%%%%%%%%%%%%%%%%%%%%%%%%%%%%%%%%%%%%%%%%%%%%%%%%%%%%
%%%%%%%%%%%%%%%%%%%%%%%%%%%%%%%%%%%%%%%%%%%%%%%%%%%%%%%%%%%%%%%%%%%%%%%%%%%%%%%

В данном разделе проводится сравнительный анализ абстрактной интерпретации и 
метода ограниченной проверки моделей. Рассаматривается классификация методов 
статического анализа. Проводится подробный обзор указанных методов статического 
анализа. На основе проведенного обзора проводится их сравнение. На основе 
проведенного анализа показывается актуальность проводимого исследования.

%%%%%%%%%%%%%%%%%%%%%%%%%%%%%%%%%%%%%%%%%%%%%%%%%%%%%%%%%%%%%%%%%%%%%%%%%%%%%%%
\section{Классификация методов статического анализа}
%%%%%%%%%%%%%%%%%%%%%%%%%%%%%%%%%%%%%%%%%%%%%%%%%%%%%%%%%%%%%%%%%%%%%%%%%%%%%%%

Статический анализ кода --- анализ программного обечпечения, производимый~(в 
отличие от динамического анализа) без реального выполнения анализируемой 
программы. Статические методы анализа ПО могут использовать для анализа не 
только исходный код программы, но и другие артефакты процесса разработки:
объектные файлы, промежуточные представления, спецификации и др. Одним из
главных недостатков методов статического анализа ПО является их вычислительная
сложность. Однако, развитие современной вычислительной техники позволяет
снизить влияние данного фактора, поэтому методы статического анализа в
настоящее время набирают популярность. 

Статические методы анализа ПО в основном разделяют на верефикацию и статический
анализ. Верификация --- это процесс математического доказательства соответствия 
программы ее исходным требованиям\footnote{Под верификацией здесь и далее 
понимается формальная верификация}. С помощью верификации можно проверить и
доказать корректность разработанной программы по отношению к формально
описанной спецификации требований. Поэтому верификация в основном направлена
на поиск функциональных ошибок. Недостатком данного метода является невозможность
его полной автоматизации: для проведения анализа требуются ручные подсказки о
промежуточных и конечных целях доказательства и начальных условиях.

Статический анализ --- это анализ программного обеспечения с целью выяснения 
каких-либо свойств программы. Он ориентирован, как правило, направлен на поиск
нефункциональных ошибок. К основным достоинствам статического анализа
относится то, что он может быть полностью автоматическим и осуществляет 
поиск полностью без участия пользователя. Недостатками данной группы подходов
являются ложные обнаружения, высокие вычислительные затраты и практически полное
отсутствие возможности поиска нефункциональных ошибок.