%%%%%%%%%%%%%%%%%%%%%%%%%%%%%%%%%%%%%%%%%%%%%%%%%%%%%%%%%%%%%%%%%%%%%%%%%%%%%%%
%%%%%%%%%%%%%%%%%%%%%%%%%%%%%%%%%%%%%%%%%%%%%%%%%%%%%%%%%%%%%%%%%%%%%%%%%%%%%%%
\chapter{Апробация разработанного прототипа и анализ полученных результатов}
\label{chapter:testing}
%%%%%%%%%%%%%%%%%%%%%%%%%%%%%%%%%%%%%%%%%%%%%%%%%%%%%%%%%%%%%%%%%%%%%%%%%%%%%%%
%%%%%%%%%%%%%%%%%%%%%%%%%%%%%%%%%%%%%%%%%%%%%%%%%%%%%%%%%%%%%%%%%%%%%%%%%%%%%%%
В данном разделе проводится оценка показателей разработанной технологии путем 
анализа результатов запуска прототипа на несколь­ких реальных программных 
проектах. Основным показателем эффективности прототипа является время его работы
в сравнении с работой системы Boralis без использования разработанной технологии.
Одним из косвенных показателей работы прототипа является количество проверок,
выполняемых SMT-решателем.

%%%%%%%%%%%%%%%%%%%%%%%%%%%%%%%%%%%%%%%%%%%%%%%%%%%%%%%%%%%%%%%%%%%%%%%%%%%%%%%
\section{Описание тестовых проектов}
%%%%%%%%%%%%%%%%%%%%%%%%%%%%%%%%%%%%%%%%%%%%%%%%%%%%%%%%%%%%%%%%%%%%%%%%%%%%%%%
Для апробации разработанного прототипа было выбрано несколько проектов с открытым
исходным кодом:
\begin{itemize}
\item библиотека \texttt{beanstalkd} --- простая и высокопроизводительная 
оче­редь, разработанная для уменьшения времени отклика при об­ращении к веб-
сервисам большого объема путем асинхронного запуска сложных задач;
\item библиотека \texttt{iputils} --- набор небольших утилит для работы с
се­тью в Lunix;
\item \texttt{clib} --- пакетный менеджер для языка C;
\item библиотека \texttt{mpc} --- легковесная библиотека парсер-комбинаторов;
\item библиотека \texttt{sds} --- простая и легковеская библиотека динамических
строк для языка C;
\item \texttt{git} — распределенная СКВ.
\end{itemize}

Данные проекты сильно отличаются друг от друга по размеру и структуре. 
\texttt{beanstalkd} является проектом среднего размера~($SLOC \approx 10k$), 
весь код является сильносвязным, его нельзя разбить на отдельные модули. Проект 
\texttt{iputils} также явля­ется средним~($SLOC \approx 19k$), однако он состоит 
из 11 маленьких подпроектов, которые практически не связаны друг с другом. 
\texttt{mpc} и \texttt{sds} --- это достаточно маленькие проекты~($SLOC \approx 
6.5k$ и $3.7k$ соответственно), которые компилируются в один файл. \texttt{clib} 
--- это библиотека среднего размера~($SLOC \approx 18k$), которая также 
компилируется в один файл. Git является большим программным проектом~($SLOC 
\approx 340k$). Проект содержит в себе большое количество модулей~($\ge 50$), которые имеют большое количество общего кода.

\begin{table}
\caption{Тестовый набор проектов}
\begin{center}
\begin{tabular}{|l|l|l|}
\hline 
\textbf{Проект} & \textbf{SLOC} & \textbf{Модульность}  \\ 
\hline 
beanstalkd & $10k$ & отсутствует \\ 
\hline 
iputils & $19k$ & явно выраженная \\ 
\hline 
git & $340k$ & явно выраженная \\ 
\hline 
clib & $18k$ & отсутствует \\ 
\hline 
mpc & $6.5k$ & отсутствует \\ 
\hline 
sds & $3.7k$ & отсутствует \\ 
\hline 
\end{tabular} 
\end{center}
\label{table:testProjects}
\end{table}

Характеристики тестовых проектов сведены в таблицу~\ref{table:testProjects}. 
Дан­ный тестовый набор позволит проверить эффективность разработан­ной технологии 
как для малых, так и для больших проектов с разной степенью функциональной 
декомпозиции.

%%%%%%%%%%%%%%%%%%%%%%%%%%%%%%%%%%%%%%%%%%%%%%%%%%%%%%%%%%%%%%%%%%%%%%%%%%%%%%%
\section{Схема работы прототипа}
%%%%%%%%%%%%%%%%%%%%%%%%%%%%%%%%%%%%%%%%%%%%%%%%%%%%%%%%%%%%%%%%%%%%%%%%%%%%%%%
Общая схема работы прототипа в тестовом режиме выглядит сле­дующим образом:
\begin{itemize}
\item для целевого проекта из СКВ получается его последняя версия;
\item в файле конфигурации проекта в качестве используемого ком­пилятора 
указывается Borealis;
\item производится сборка проекта;
\item производится анализ результатов работы прототипа.
\end{itemize}

%%%%%%%%%%%%%%%%%%%%%%%%%%%%%%%%%%%%%%%%%%%%%%%%%%%%%%%%%%%%%%%%%%%%%%%%%%%%%%%
\section{Результаты тестирования}
%%%%%%%%%%%%%%%%%%%%%%%%%%%%%%%%%%%%%%%%%%%%%%%%%%%%%%%%%%%%%%%%%%%%%%%%%%%%%%%
Главной характеристикой, которая анализировалась при тестировании прототипа, 
является время работы системы на тестовых проектах в сравнении с временем
работы Borealis без модуля АИ. Также, при апробации анализировалось количество 
проверок, которые выполнял SMT-решатель. Результаты апробации прототипа 
приведены в таблицах~\ref{table:timeResults}~(временные показатели) и~\ref{table:checkResults}~(качественные показатели).

\begin{table}
\caption{Временные показатели работы прототипа}
\centering
\begin{tabular}{|r|c|c|c|}
\hline
           & АИ LLVM IR   & АИ PS        & без АИ    \\ \hline
beanstalkd & 8.935s       & 8.934s       & 8.969s    \\ \hline
clib       & 2m45.624s    & 44.746s      & 43.213s   \\ \hline
git        & 53m39.379s   & 45m16.965s   & 48m9.496s \\ \hline
iputils    & 52.555s      & 53.958s      & 56.239s   \\ \hline
mpc        & 22.379s      & 20.966s      & 21.067s   \\ \hline
sds        & 4.660s       & 5.365s       & 5.721s    \\ \hline
\end{tabular}
\label{table:timeResults}
\end{table}

Проанализировав результаты в таблице~\ref{table:timeResults}, можно сделать 
следующие выводы:
\begin{itemize}
\item интерпретация LLVM IR для большинства проектов оказывается медленнее, чем
анализ без АИ;
\item интерпретация PS для большинства проектов оказывается быстрее, чем анализ
без АИ;
\item прирост в скорости не превышает $6\%$~(а для большинства проектов прирост
еще меньше).
\end{itemize}

Можно заметить, что на проектах \texttt{clib} и \texttt{git} прототип с 
использованием интерпретации LLVM IR оказывается значительно медленнее, чем
Borealis без АИ. Это объясняется следующими причинами. В проекте \texttt{clib}
создается большое количество глобальных массивов большого размера, которые
затем активно используются во всех функциях проекта. Каждый раз при обращении
к какому либо из массивов АИ приходится проделать большое количество операций
объединения доменов, что сильно сказывается на производительности.

Как упоминалось ранее, проект \texttt{git} состоит из множества модулей, 
которые имеют очень большой объем общего кода. Поэтому, одни и те же функции
многократно анализируются в каждом модуле, что увеличивает время работы 
прототипа. Система Borealis умеет эффективно обрабатывать подобные случаи: она
сохраняет информацию об уже обработанных функциях для каждого проекта и не 
перезапускает анализ функции если она уже проанализирована в другом модуле. 
Интерпретация на уровне PS тоже обрабатывает подобные ситуации: интерпретация 
PS запускается только после того, как Borealis убедился, что конкретный PS 
необходимо проанализировать.

\begin{table}
\caption{Количественные показатели работы прототипа}
\centering
\footnotesize
\begin{tabular}{|r|c|c|c|c|c|c|}
\hline
            \multirow{2}{*}{}
           & \multicolumn{2}{c|}{АИ LLVM IR} 
           & \multicolumn{2}{c|}{АИ PS} 
           & \multicolumn{2}{c|}{без АИ} \\ \cline{2-7}
           & SAT    & UNSAT   & SAT    & UNSAT   & SAT    & UNSAT   \\ \hline
beanstalkd & 356    & 252     & 354    & 161     & 360    & 247     \\ \hline
clib       & 599    & 258     & 959    & 234     & 960    & 449     \\ \hline
git        & 13177  & 10964   & 19402  & 9720    & 19665  & 15222   \\ \hline
iputils    & 519    & 1786    & 534    & 934     & 543    & 1819    \\ \hline
mpc        & 474    & 305     & 497    & 146     & 506    & 286     \\ \hline
sds        & 149    & 84      & 170    & 40      & 170    & 109     \\ \hline
\end{tabular}
\label{table:checkResults}
\end{table}