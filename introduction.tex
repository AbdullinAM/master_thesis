%%%%%%%%%%%%%%%%%%%%%%%%%%%%%%%%%%%%%%%%%%%%%%%%%%%%%%%%%%%%%%%%%%%%%%%%%%%%%%%%
\intro
%%%%%%%%%%%%%%%%%%%%%%%%%%%%%%%%%%%%%%%%%%%%%%%%%%%%%%%%%%%%%%%%%%%%%%%%%%%%%%%%
В современном мире программное обеспечение~(ПО) внедрилось во все сферы 
человеческой жизни. В связи с этим, проблема проверки и обеспечения качества 
программного обеспечения имеет все большее значение. Одним из основных способов 
проверки качества ПО является анализ программ. Однако, анализ программ также 
cталкивается с множеством проблем, наиболее важными из которых являются 
производительность и качество анализа. Под качеством анализа понимается его 
точность (отношение числа корректно найденных дефектов к общему числу найденных 
де­фектов) и полнота (отношение числа корректно найденных дефектов к общему 
числу дефектов в программе). Эти две проблемы напрямую связаны между собой:
чем выше качество проводимого анализа, тем больше вычислительных ресурсов он 
потребует.

Можно выделить две основные группы методов анализа ПО. Первая группа подходов 
позволяет провести анализ программы очень быстро, но, соответственно, сильно
ограничена по возможностям. Представителем данной группы подходов является 
сигнатурный анализ ПО. Вторая группа подходов позволяет провести более 
качественный и детальный анализ, но обладает высокой вычислительной сложностью.
Классическим представителем второй группы подходов является метод ограниченной 
проверки моделей~(bounded model checking, BMC)~\cite{bmc}. В приведенном 
разделении отдельно стоит выделить абстрактную интерпретацию~(abstract 
interpretation, AI)~\cite{ai}. Ее незльзя однозначно отнести ни к одной 
из описанных групп, потому что AI позволяет адаптировать свою вычислительную 
сложность и качество анализа.

Целью данной работы является исследование возможности объединения методов 
абстрактной интерпретации и ограниченной проверки моделей. Основная идея 
данного исследования заключается в том, что благодаря объединению двух 
указанных методов, можно добиться повышения производительности анализа. 
Проводимые в данной работе исследования планируется проводить на основе 
инструмента статического анализа Borealis~\cite{borealis}.

Работа состоит из 5 разделов. В разделе 1 проводится сравнительный анализ 
методов абстрактной интерпретации и ограниченной проверки моделей. На основе 
проведенного анализа показывается актуальность проведения описываемого 
исследования.

Раздел 2 посвящен постановке задачи объединения абстрактной интерпретации и
метода ограниченной проверки моделей. Ставятся требования к разрабатываемой 
технологии. В данном разделе также ставится задача разработки модуля AI для
системы Borealis.

В 3 разделе рассматриваются модели представления кода и описывается предлагаемый 
алгоритм объединения AI и BMC в рамках выбранной модели кода. В данном разделе
также проводится анализ существующих решений в области AI языка C.

В разделе 4 описывается разработка модуля абстрактной интерпретации для системы 
статического анализа Borealis. В разделе описывается архитектура описываемого 
модуля, описываются выбранные инструменты и библиотеки, используемые при 
разработке.

Раздел 5 посвящен апробации разработанного прототипа на реальных программных 
проектах. Представляются результаты апробации и показывается применимость
предлагаемого подхода на реальных програмных проектах.