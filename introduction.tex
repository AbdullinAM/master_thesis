%%%%%%%%%%%%%%%%%%%%%%%%%%%%%%%%%%%%%%%%%%%%%%%%%%%%%%%%%%%%%%%%%%%%%%%%%%%%%%%%
\intro
%%%%%%%%%%%%%%%%%%%%%%%%%%%%%%%%%%%%%%%%%%%%%%%%%%%%%%%%%%%%%%%%%%%%%%%%%%%%%%%%

В современном мире программное обеспечение внедрилось во все сферы человеческой 
жизни. В связи с этим, проблема проверки и обеспечения качества программного 
обеспечения встает все более остро(??). Одним из основных способов проверки 
качества ПО является анализ программ. Однако, анализ программ также 
cталкивается с множеством проблем, наиболее важными из которых являются 
производительность и качество анализа. Под качеством анализа понимается его 
точность (отношение числа корректно найденных дефектов к общему числу найденных 
де­фектов) и полнота (отношение числа корректно найденных дефектов к общему 
числу дефектов в программе). Эти две проблемы являются в некотором роде 
обратными: для того, чтобы провести качественный анализ программы, необходимо 
потратить большое количество вычислительных ресурсов.

Различные виды анализа можно разделить на две основные группы, которые по-
разному пытаются найти баланс между этими проблемами. Первая группа подходов жертвует качеством анализа для повышения его производительности. Представителем 
данной группы подходов является сигнатурный анализ (ссылка). Вторая группа 
подходов жертвует производительностью анализа ради повышения его качетсва. 
Классическим представителем второй группы подходов является метод ограниченной 
проверки моделей~\cite{bmc}. Примечетальной в данной классификации является 
абстрактная интерпретация~\cite{ai}. Ее незльзя однозначно отнести ни к одной 
из этих двух групп, потому что данный метод позволяет варьировать свою 
вычислительную сложность и качество анализа.

Целью данной работы является исследование возможности объединения методов 
абстрактной интерпретации и ограниченной проверки моделей. Основная идея 
данного исследования заключается в том, что благодаря объединению двух 
указанных методов, можно добиться повышения производительности анализа. 
Проводимые в данной работе исследования планируется проводить на основе 
инструмента статического анализа Borealis~\cite{borealis}.

Работа состоит из 6 разделов. В разделе 1 проводится сравнительный анализ 
методов абстрактной интерпретации и ограниченной проверки моделей. На основе 
проведенного анализа показывается актуальность проведения описываемого 
исследования.

Раздел 2 посвящен анализу существующих решений в области абстрактной 
интерпретации. Рассматриваются существующие фреймворки абстрактной 
интерпретации, описываются их преимущества и недостатки.

В 3 разделе приводится постановка задачи данного исследования. Выбирается 
целевой язык программирования, ключевые параметры технологии. В данном разделе 
также ставится задача разработки модуля абстрактной интерпретации для системы 
статического анализа Borealis.

Раздел 4 посвящен описанию ...

В разделе 5 описывается разработка модуля абстрактной интерпретации для системы 
статического анализа Borealis. В разделе описывается архитектура описываемого 
модуля, описываются выбранные инструменты и библиотеки, используемые при 
разработке.

В 6 разделе проводится апробация созданного прототипа. Показывается ....