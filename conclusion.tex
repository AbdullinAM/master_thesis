%%%%%%%%%%%%%%%%%%%%%%%%%%%%%%%%%%%%%%%%%%%%%%%%%%%%%%%%%%%%%%%%%%%%%%%%%%%%%%%%
\conclusion
%%%%%%%%%%%%%%%%%%%%%%%%%%%%%%%%%%%%%%%%%%%%%%%%%%%%%%%%%%%%%%%%%%%%%%%%%%%%%%%%
Одной из главных проблем статического анализа является его высокая 
вычислительная сложность. Одними из наиболее популярных методов СА являются 
абстрактная интерпретация и метод ограниченной проверки моделей.
В данной работе рассматривается подход к объединению указанных методов с целью
повышения скорости анализа: использовать AI для исключения лишних проверок в BMC.

В работе выполнен анализ предметной области и сравнение абстрактной интерпретации
и метода ограниченной проверки моделей. Также в работе выполнен анализ 
существующих решений в области абстрактной интерпретации языка C. Показана 
актуальность разработки технологии объединения AI и BMC.

Основная идея предложенного подхода заключается в том, чтобы использовать 
абстрактную интерпретацию для исключения лишних проверок в методе ограниченной 
проверки моделей: благодаря тому, что AI обладает полнотой, она позволяет 
исключить те проверки в BMC, которые гарантированно не сообщат об ошибке. 
Предложенный подход использует AI на двух уровнях представления программы:
на уровне исходного кода и на уровне упрощенного представления программы, 
используемого в BMC. В работе представлено описание всех этапов предлагаемого
подхода: используемые абстрактные домены, алгоритмы интерпретации, алгоритмы 
поиска ошибок по результатам интерпретации.

На базе системы статического анализа Borealis был разработан прототип, 
реализующий предложенную технологию. Была проведена апробация разработанного
прототипа на нескольких реальных проектах различного объема и сложности. 
Апробация показала, что предложенный подход позволяет повысить скорость анализа,
однако прирост к скорости не очень большой. Также, апробация показала, что
AI на уровне LLVM IR повышает точность анализа: она позволяет уменьшить
количество ложных срабатываний~(false positives) анализа.

Дальнейшее развитие результатов может выполняться в нескольких направлениях:
\begin{itemize}
\item исследование возможности использования других~(более точных) абстрактных
доменов;
\item реализация возможности передачи информации о проинтерпретированных функциях
между разными модулями программы.
\end{itemize}